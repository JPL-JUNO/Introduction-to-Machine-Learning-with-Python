\chapter{处理文本数据}
文本数据通常被表示为由字符组成的字符串。在上面给出的所有例子中,文本数据的长度
都不相同。这个特征显然与前面讨论过的数值特征有很大不同,我们需要先处理数据,然
后才能对其应用机器学习算法。
\section{用字符串表示的数据类型}
文本通常只是数据集中的字符串,但并非所有的字符串特征都应该被
当作文本来处理。
可能会遇到四种类型的字符串数据:
\begin{itemize}
    \item 分类数据
    \item 可以在语义上映射为类别的自由字符串
    \item 结构化字符串数据
    \item 文本数据
\end{itemize}

分类数据(categorical data)是来自固定列表的数据。比如你通过调查人们最喜欢的颜色
来收集数据,你向他们提供了一个下拉菜单,可以从“红色”“绿色”“蓝色”“黄色”“黑
色”“白色”“紫色”和“粉色”中选择。这样会得到一个包含 8 个不同取值的数据集,
这 8 个不同取值表示的显然是分类变量。你可以通过观察来判断你的数据是不是分类数
据(如果你看到了许多不同的字符串,那么不太可能是分类变量),并通过计算数据集中
的唯一值并绘制其出现次数的直方图来验证你的判断。

现在想象一下,你向用户提供的不是一个下拉菜单,而是一个文本框,让他们填写自己最
喜欢的颜色。许多人的回答可能是像“黑色”或“蓝色”之类的颜色名称。其他人可能会
出现笔误,使用不同的单词拼写(比如“gray”和“grey”),或使用更加形象的具体名称
(比如“午夜蓝色”)。

从
文本框中得到的回答属于上述列表中的第二类,可以在语义上映射为类别的自由字符串
(free strings that can be semantically mapped to categories)。可能最好将这种数据编码为分类
变量,你可以利用最常见的条目来选择类别,也可以自定义类别,使用户回答对应用有意
义。这样你可能会有一些标准颜色的类别,可能还有一个“多色”类别(对于像“绿色与
红色条纹”之类的回答)和“其他”类别(对于无法归类的回答)。这种字符串预处理过
程可能需要大量的人力,并且不容易自动化。如果你能够改变数据的收集方式,那么我们
强烈建议,对于分类变量能够更好表示的概念,不要使用手动输入值。

通常来说,手动输入值不与固定的类别对应,但仍有一些内在的结构(structure),比如地
址、人名或地名、日期、电话号码或其他标识符。这种类型的字符串通常难以解析,其处
理方法也强烈依赖于上下文和具体领域。

最后一类字符串数据是自由格式的文本数据(text data),由短语或句子组成。例子包括
推文、聊天记录和酒店评论,还包括莎士比亚文集、维基百科的内容或古腾堡计划收集
的 50 000 本电子书。在文本分析的语境中,数据集通
常被称为语料库(corpus)\marginpar{语料库},每个由单个文本表示的数据点被称为文档\marginpar[文档]{文档}(document)。这
些术语来自于信息检索(information retrieval,IR)和自然语言处理(natural language
processing,NLP)的社区,它们主要针对文本数据。
\section{示例应用:电影评论的情感分析}
\section{将文本数据表示为词袋}
用于机器学习的文本表示有一种最简单的方法,也是最有效且最常用的方法,就是使用词
袋(bag-of-words)表示。使用这种表示方式时,我们舍弃了输入文本中的大部分结构,如
章节、段落、句子和格式,只计算语料库中每个单词在每个文本中的出现频次。

对于文档语料库,计算词袋表示包括以下三个步骤:
\begin{enumerate}
    \item 分词(tokenization)。将每个文档划分为出现在其中的单词 [ 称为词例(token)],比如
          按空格和标点划分。
    \item 构建词表(vocabulary building)。收集一个词表,里面包含出现在任意文档中的所有词,
          并对它们进行编号(比如按字母顺序排序)。
    \item 编码(encoding)。对于每个文档,计算词表中每个单词在该文档中的出现频次。
\end{enumerate}
\subsection{将词袋应用于测试数据集}
词袋表示是在 CountVectorizer 中实现的,它是一个变换器(transformer)。
\subsection{将词袋应用于电影评论}
\section{停用词}
\section{用tf-idf缩放数据}
\section{研究模型系数}
\section{多个单词的词袋(n元分词)}
\section{高级分词、词干提取与词形还原}
\section{主题建模与文档聚类}
\subsection{隐含狄利克雷分布}