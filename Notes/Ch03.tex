\chapter{无监督学习与预处理\label{chapter03}}
无监督学习包括没有已知输出、没有老师指导学习算法的各种机器学习。在无监督学习中,学习算法只有输入数据,并需要从这些数据中提取知识。
\section{无监督学习的类型}
这里将研究两种类型的无监督学习:数据集变换与聚类。

数据集的\textbf{无监督变换}(unsupervised transformation)是创建数据新的表示的算法,与数据的原始表示相比,新的表示可能更容易被人或其他机器学习算法所理解。无监督变换的一个常见应用是降维(dimensionality reduction),它接受包含许多特征的数据的高维表示,并找到表示该数据的一种新方法,用较少的特征就可以概括其重要特性。降维的一个常见应用是为了可视化将数据降为二维。

无监督变换的另一个应用是找到“构成”数据的各个组成部分。这方面的一个例子就是对文本文档集合进行主题提取。

\textbf{聚类算法}(clustering algorithm)将数据划分成不同的组,每组包含相似的物项。
\section{无监督学习的挑战}
无监督学习的一个主要挑战就是评估算法是否学到了有用的东西。无监督学习算法一般用于不包含任何标签信息的数据,所以我们不知道正确的输出应该是什么。因此很难判断一个模型是否“表现很好”。通常来说,评估无监督算法结果的唯一方法就是人工检查。

因此,如果数据科学家想要更好地理解数据,那么无监督算法通常可用于探索性的目的,而不是作为大型自动化系统的一部分。无监督算法的另一个常见应用是作为监督算法的预处理步骤。学习数据的一种新表示,有时可以提高监督算法的精度,或者可以减少内存占用和时间开销。

虽然预处理和缩放通常与监督学习算法一起使用,但缩放方法并没有用到与“监督”有关的信息,所以它是无监督的。
\section{预处理与缩放}
一些算法(如神经网络和 SVM)对数据缩放非常敏感。因此,通常的做法是对特征进行调节,使数据表示更适合于这些算法。通常来说,这是对数据的一种简单的按特征的缩放和移动。\autoref{Different ways to rescale and preprocess a dataset} 给出了一个简单的例子:

\figures{Different ways to rescale and preprocess a dataset}
\subsection{不同类型的预处理}
\subsection{应用数据变换}
\subsection{对训练数据和测试数据进行相同的缩放}
\subsection{预处理对监督学习的作用}
\section{降维、特征提取与流形学习}
\subsection{主成分分析}
\subsection{非负矩阵分解}
\subsection{用t-SNE进行流形学习}
\section{聚类}
\subsection{k均值聚类}
\subsection{凝聚聚类}
\subsection{DBSCAN}
\subsection{聚类算法的对比与评估}
\subsection{聚类方法小结}
\section{小结与展望}