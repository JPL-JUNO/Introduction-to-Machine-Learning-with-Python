\begin{table}
    \centering
    \caption{一些样本数据集}
    \begin{tabular}{cccl}
        \hline
        名称     & 来源 & 特征量            & 用途      \\
        \hline
        forge  & 模拟 & $26\times 2$   & 二分类数据集  \\
        wave   & 模拟 & $40\times 1$   & 回归算法数据集 \\
        cancer & 真实 & $569\times 30$ & 未指明     \\
        boston & 真实 & $506\times 13$ & 未指明     \\
        \hline
    \end{tabular}
\end{table}

\begin{table}
    \centering
    \caoption{一些缩放方式}
    \begin{tabularx}{\textwidth}{ccX}
        \hline
        缩放方式           & 作用位置 & 描述                                                                                                                                                              \\
        \hline
        StandardScaler & 特征   & 确保每个特征的平均值为 0、方差为 1,使所有特征都位于同一量级。但这种缩放不能保证特征任何特定的最大值和最小值。                                                                                                       \\
        RobustScaler   & 特征   & 与StandardScaler 类似,确保每个特征的统计属性都位于同一范围。但 RobustScaler 使用的是中位数和四分位数 1,而不是平均值和方差。这样 RobustScaler 会忽略与其他点有很大不同的数据点(比如测量误差)。这些与众不同的数据点也叫异常值(outlier),可能会给其他缩放方法造成麻烦。 \\
        MinMaxScaler   & 特征   & 移动数据,使所有特征都刚好位于 0 到 1 之间。                                                                                                                                       \\
        Normalizer     & 样本   & 它对每个数据点进行缩放,使得特征向量的欧式长度等于 1。换句话说,它将一个数据点投射到半径为 1 的圆上(对于更高维度的情况,是球面)。这意味着每个数据点的缩放比例都不相同(乘以其长度的倒数)。如果只有数据的方向(或角度)是重要的,而特征向量的长度无关紧要,那么通常会使用这种归一化。                  \\
        \hline
    \end{tabularx}
\end{table}

\begin{table}[H]
    \begin{tabular}{ll}
        \hline
        \multicolumn{2}{c}{\verb|estimator.fit(X_train, [y_train])|}          \\
        \verb|estimator.predict(X_test)| & \verb|estimator.transform(X_test)| \\
        \hline
        Classification                   & Preprocessing                      \\
        Regression                       & Dimensionality Reduction           \\
        Clustering                       & Feature Extraction                 \\
                                         & Feature Selection                  \\
        \hline
    \end{tabular}
\end{table}

