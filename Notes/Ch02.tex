\chapter{监督学习}
\section{分类与回归}
监督机器学习问题主要有两种,分别叫作分类(classification)与回归(regression)。

分类问题的目标是预测类别标签(class label),这些标签来自预定义的可选列表。分类问题有时可分为二分类
(binary classification,在两个类别之间进行区分的一种特殊情况)和多分类(multiclass classification,在两个以上的类别之间进行区分)。在二分类问题中,我们通常将其中一个类别称为正类(positive class),另一个类别称为反类(negative class)。这里的“正”并不代表好的方面或正数,而是代表研究对象。

回归任务的目标是预测一个连续值,编程术语叫作浮点数(floating-point number),数学术语叫作实数(real number)。


区分分类任务和回归任务有一个简单方法,就是问一个问题:输出是否具有某种连续性。如果在可能的结果之间具有连续性,那么它就是一个回归问题。

\section{泛化、过拟合与欠拟合}
在监督学习中,我们想要在训练数据上构建模型,然后能够对没见过的新数据(这些新数据与训练集具有相同的特性)做出准确预测。如果一个模型能够对没见过的数据做出准确预测,我们就说它能够从训练集\textbf{泛化}(generalize)到测试集。我们想要构建一个泛化精度尽可能高的模型。

判断一个算法在新数据上表现好坏的唯一度量,就是在测试集上的评估。然而从直觉上看\footnote{在数学上也可以证明这一点。},我们认为简单的模型对新数据的泛化能力更好。因此,我们总想找到最简单的模型。构建一个对现有信息量来说过于复杂的模型,这被称为\textbf{过拟合}(overfitting)。如果你在拟合模型时过分关注训练集的细节,得到了一个在训练集上表现很好、但不能泛化到新数据上的模型,那么就存在过拟合。与之相反,如果你的模型过于简单,那么你可能无法抓住数据的全部内容以及数据中的变化,你的模型甚至在训练集上的表现就很差。选择过于简单的模型被称为\textbf{欠拟合}(underfitting)。

\figures{Trade-off of model complexity against training and test accuracy}
\subsection*{模型复杂度与数据集大小的关系}
需要注意,模型复杂度与训练数据集中输入的变化密切相关:数据集中包含的数据点的变化范围越大,在不发生过拟合的前提下你可以使用的模型就越复杂。通常来说,收集更多的数据点可以有更大的变化范围,所以更大的数据集可以用来构建更复杂的模型。但是,仅复制相同的数据点或收集非常相似的数据是无济于事的。

收集更多数据,适当构建更复杂的模型,对监督学习任务往往特别有用。本书主要关注固定大小的数据集。在现实世界中,你往往能够决定收集多少数据,这可能比模型调参更为有效。永远不要低估更多数据的力量!

\section{监督学习算法}
现在开始介绍最常用的机器学习算法,并解释这些算法如何从数据中学习以及如何预测。我们还会讨论每个模型的复杂度如何变化,并概述每个算法如何构建模型。我们将说明每个算法的优点和缺点,以及它们最适应用于哪类数据。此外还会解释最重要的参数和选项的含义,许多算法都有分类和回归两种形式。

\subsection{一些样本数据集}
我们将使用一些数据集来说明不同的算法。其中一些数据集很小,而且是模拟的,其目的是强调算法的某个特定方面。其他数据集都是现实世界的大型数据集。

\begin{table}
    \centering
    \caption{一些样本数据集}
    \begin{tabular}{cccl}
        \hline
        名称     & 来源 & 特征量            & 用途      \\
        \hline
        forge  & 模拟 & $26\times 2$   & 二分类数据集  \\
        wave   & 模拟 & 2              & 回归算法数据集 \\
        cancer & 真实 & $569\times 30$ & 未指明     \\
        boston & 真实 & $506\times 13$ & 未指明     \\
        \hline
    \end{tabular}
\end{table}