\chapter{模型评估与改进}
为了评估我们的监督模型,我们使用 \verb|train_test_split| 函数将数据集划分为训练集和测试集,在训练集上调用 fit 方法来构建模型,并且在测试集上用 score 方法来评估这个模型——对于分类问题而言,就是计算正确分类的样本所占的比例。

请记住,之所以将数据划分为训练集和测试集,是因为我们想要度量模型对前所未见的新数据的泛化性能。我们对模型在训练集上的拟合效果不感兴趣,而是想知道模型对于训练过程中没有见过的数据的预测能力。

本章我们将从两个方面进行模型评估。我们首先介绍交叉验证,然后讨论评估分类和回归性能的方法,其中前者是一种更可靠的评估泛化性能的方法,后者是在默认度量(score方法给出的精度和 $R^2$)之外的方法。

我们还将讨论网格搜索,这是一种调节监督模型参数以获得最佳泛化性能的有效方法。
\section{交叉验证}
交叉验证(cross-validation)是一种评估泛化性能的统计学方法,它比单次划分训练集和测试集的方法更加稳定、全面。在交叉验证中,数据被多次划分,并且需要训练多个模型。最常用的交叉验证是 k 折交叉验证(k-fold cross-validation),其中 k 是由用户指定的数字,通常取 5 或 10。在执行 5 折交叉验证时,首先将数据划分为(大致)相等的 5 部分,每一部分叫作折(fold)。接下来训练一系列模型。使用第 1 折作为测试集、其他折(2~5)作为训练集来训练第一个模型。利用 2~5 折中的数据来构建模型,然后在 1 折上评估精度。之后构建另一个模型,这次使用 2 折作为测试集,1、3、4、5 折中的数据作为训练集。利用 3、4、5 折作为测试集继续重复这一过程。对于将数据划分为训练集和测试集的这 5 次划分,每一次都要计算精度。最后我们得到了 5 个精度值。
\figures{Data splitting in five-fold cross-validation}
\subsection{scikit-learn中的交叉验证}
scikit-learn 是利用 \verb|model_selection| 模块中的 \verb|cross_val_score| 函数来实现交叉验证的。\verb|cross_val_score| 函数的参数是我们想要评估的模型、训练数据与真实标签。默认情况下,\verb|cross_val_score| 执行 5 折交叉验证,返回 5 个精度值。可以通过修改 cv 参数来改变折数。

总结交叉验证精度的一种常用方法是计算平均值。
\subsection{交叉验证的优点}
\subsection{分层k折交叉验证和其他策略}
\subsubsection{对交叉验证的更多控制}
\subsubsection{留一法交叉验证}
\subsubsection{打乱划分交叉验证}
\subsubsection{分组交叉验证}
\section{网格搜索}
\subsection{简单网格搜索}
\subsection{参数过拟合的风险与验证集}
\subsection{带交叉验证的网格搜索}
\subsubsection{分析交叉验证的结果}
\subsubsection{在非网格的空间中搜索}
\subsubsection{使用不同的交叉验证策略进行网格搜索}
\section{评估指标与评分}
\subsection{牢记最终目标}
\subsection{二分类指标}
\subsubsection{错误类型}
\subsubsection{不平衡数据集}
\subsubsection{混淆矩阵}
\subsubsection{考虑不确定性}
\subsubsection{准确率-召回率曲线}
\subsubsection{受试者工作特征(ROC)与AUC}
\subsection{多分类指标}
\subsection{回归指标}
\subsection{在模型选择中使用评估指标}
\section{小结}